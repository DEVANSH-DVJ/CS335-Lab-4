\documentclass[12pt, fleqn]{article}

\usepackage[left=0.75in, right=0.75in, bottom=0.75in, top=1.0in]{geometry}
\usepackage{amsmath}
\usepackage{amssymb}
\usepackage{amsthm}
\usepackage{mathtools}
\usepackage{hyperref}
\usepackage{ulem}
\usepackage{enumitem}
\usepackage{floatrow}
\usepackage{graphicx}
\usepackage[export]{adjustbox}
\usepackage{sectsty}
\usepackage{enumitem}
\renewcommand{\thesubsubsection}{(\roman{subsubsection})}

\usepackage[dvipsnames]{xcolor}
\usepackage[perpage]{footmisc}

\usepackage{fancyhdr}
\pagestyle{fancy}
\fancyhf{}
\lhead{190100044}
\rhead{Assignment 4}
\renewcommand{\footrulewidth}{1.0pt}
\cfoot{Page \thepage}

\setlength{\parindent}{0em}

\title{Assignment 4}
\author{Devansh Jain, 190100044}
\date{30th Oct 2021}

\DeclareMathOperator*{\argmax}{arg\,max}
\DeclareMathOperator*{\argmin}{arg\,min}

\begin{document}

% \pagenumbering{gobble}
\maketitle
\tableofcontents
\thispagestyle{empty}
\setcounter{page}{0}

\newpage
\section{Clustering}
\subsection{CS 335 KMeans Implementation}



\newpage
\section{Kernel design and Kernelized clustering}
\subsection{CS 337: Proving Kernel Validity}
We are going to use the following property of kernels from Lecture slides (Lecture 11):
\begin{equation*}
    \begin{aligned}
        K(x,y) = \sum_{d=1}^r \alpha_d(x^T y)^d \text{ where } \alpha_d \ge 0 \text{ is a kernel } (r \text{ can be } \infty)
    \end{aligned}
\end{equation*}

Another property of kernels which we are going to use is:
\begin{equation*}
    \begin{aligned}
         & K(x, y) \text{ is a kernel } \implies K'(x, y) = f(x) f(y) K(x, y) \text{ is also a kernel} \\
         & \text{Corresponding feature map, } \phi'(x) := f(x) \phi(x)
    \end{aligned}
\end{equation*}


An important property of exponential function which we are going to exploit is:
\begin{equation*}
    \begin{aligned}
        \exp(x) = \sum_{k=0}^{\infty} \frac{x^k}{k!}
    \end{aligned}
\end{equation*}

Using the above properties, we can conclude that $\exp(\alpha x^T y)$ where $\alpha \ge 0$ is a kernel.

Now, if take $\alpha = \dfrac{1}{\sigma^2}$ and $f(x) = \exp\bigg(- \dfrac{x^T x}{2 \sigma^2}\bigg)$, we get: \\
$K_\alpha (x, y) = \exp\bigg(- \dfrac{||x - y||^2}{2 \sigma^2}\bigg) = \exp\bigg(- \dfrac{x^T x}{2 \sigma^2}\bigg) \exp\bigg(- \dfrac{y^T y}{2 \sigma^2}\bigg) \exp\bigg(\dfrac{x^T y}{\sigma^2}\bigg)$ is a kernel.

Hence, proved. \hfill $\qed$


\subsection{CS 337: Simple Kernel Design}
\subsubsection{}

\subsubsection{}



\end{document}
